\documentclass{article}
\usepackage[utf8]{inputenc}
\usepackage[subpreambles=true]{standalone}

\usepackage{verbatim}

\begin{document}

\title{Final Report}
\date{}

\section*{Introduction}

\section*{Background}

\subsection*{Technology Employed}

JavaScript is used for the front end of the game, it renders the game world and the action that takes place in it within the browser. JavaScript enables for the creation of interactive content and control of the multimedia available from the browser. Its' capabilities from runtime object construction to dynamic script evaluation allows for concise and effective development. Such properties are all ideal for the making of a game. 

Two frameworks of Javascript in particular are used for the project, Angular JS and Pixi JS. 


 
AngularJS provides a structural framework for designing dynamic web applications through extending HTML's syntax, such that components of application can be represented with data binding and dependency injection. This allows for a template to be built for the project, where the content can be presented consistently. New components can also be added to the existing structure of the application easily through the utilised template.




Pixi JS provides a framework for rendering graphics with WebGL support or the HTML5’s canvas. It is a better alternative to Adobe Flash that could design quality material effectively and with ease, especially with Flash's decrease of a presence on the web. It achieves this by avoiding the focus on the low level code required to create interactive content. There are no worries of browser or device inconsistencies by allowing for cross platform compatibility and graceful degradation.







Maven is used as the build tool for the game. It provides an uniform build system that aims to provide a build process that is easy to use, that provides crucial information regarding the project, and the capacity to update features of the project with transparency. The uniform system is useful for joining all the components of the back end together to formulate the game world of the project.







Scala is the primary language used to write the back end of the project. Scala incorporates both object oriented and functional programming, allowing the code of complex operations such as recursions or complicated structures to be simplified. Operations in the language are concise and can be combined with ease. It is statically typed which prevents misuses and errors from the collection of operations.This is chosen for the project as it has the capacity to simplify the coding for a lot of the more complex mechanics. Being statically typed also helps to identify any major fault from development during compilation, this reduces the risk of any game breaking bug making through to the final product.  







Insomnia is used to test out the JSON querying from the game server. 












\label{Strategy_Games}

\subsection*{Strategy Games}

For each type of game, there exists a core mechanic. This is a set of actions within the game that prioritise the rules of the game in order to have a greater chance of winning. The core mechanic of strategy games is to plan better than your opponents through utilising the available resources and options in the game to your advantage. Despite of the same core mechanic, there exist many different types of strategy games, real time or turn based, strategy or tactic based, war or empire simulator, etc. 

This project focuses on 4x strategy games. 4x stands for explore, expand, exploit and exterminate. These are the main concepts of the game that formulate different mechanics that intertwine with each other in conforming the core mechanic.


Exploration is about gathering information concerning the game world, such as where the other players are located, where are the points of interest in terms of resources or other strategic advantages. By having knowledge about the terrain and the surrounding, this allows the player to make more informed decisions and better planning.

Expansion makes use of opportunities for scaling up the existing amenities and production. The focus is to take advantage of the increased capacity in the existing mechanism to make more opportunities in the future. This allows the player to be equipped with more resources and information, which enables the player to continue expansion in the future

Exploitation is concerned with grasping the other mechanism of the game, in order to make full use of the systems in place. For many games, this comes in the form of technology tree upgrades that could improve the efficiency of current productions. But it could be anything that could be altered by the player through careful planning, to fully realise the potential of the mechanism in place. 

Extermination is as the name suggests, erasing the existence or significance of other players. This does not necessarily mean that combat mechanics are always required. Alternatives are actions that could render the impact of other players redundant, for example, in ..., there are ... victories that allow a player to win through completing ... And by finishing ... before others, this will win the game, which makes everything that the other players have done insignificant.  

Despite the differences between the 4 concepts, they are all related. For example, without exploration the player will not be able to make the best out of expansion. Exploration make information available to be exploited. Expansion is also a form of exploitation of the game mechanic to extend the current resources. By constantly expanding, the player will be forced to confront other players where elements of extermination comes into play. As there can only be one victor in the end.

Together, the 4 concepts create a rich environment of content for the game, where the different mechanics develop differently as the game progress throughout the 3 stages: start of the game, mid game, and late game.

At the start of the game, the focus is on exploration and using the nearby environment or the resources available to create a foundation. This will be built upon over time, switching the focus onto expansion and exploitation to unlock different components from the mechanics of the game. This allows players to begin their strategy on achieving victory. 


The focus of the mid game is primarily on exploitation, where players begin executing their strategies in full effect through the foundation they have been building from the start of the game. This is also when the players are most likely going to clash with each other from expansion, which could change the focus of some towards extermination.

In the late game, majority of the players will have their strategies well established, such that the focus from the mid game is carried over to the late game as well. A common issue in the late game is the snowball effect, this is when the game becomes a waiting game due to a substantial difference between one player with the others. That one particular player possess such an excess of resources or control over the game world, it becomes only a matter of time for that player to win.

For a lot of strategy games, the late game is the least entertaining part of the game. This is because resources and amenities will have been very well established from the early parts of the game, such that the value of them have lost significance. They become something that players will always have at their disposal, so the focus becomes more about management. Also, the different elements of the game and components of the mechanics would have been fully explored. The lack of content that could be offered at that point is a huge contrast to the early and mid game. 











\begin{comment}



\end{comment}


\section*{Group Management}

Business Analyst is the name of Jeremy's official role within this group project. The role includes handling the communication between the group, the supervisor and the customer. He makes sure meetings are arranged regularly, such that the supervisor and the customer are informed of the progress for the project.

Apart from the administrative tasks, he also contribute toward the development of the project. 

He has been responsible for researching relevant material for the development of the game.

Looking into aspects of strategy games such as the 4 main objectives: Explore, Exploit, Expand and Exterminate. The mechanics and balance of the Technology Tree. The stages and life cycle of a strategy game, etc.

He also researched about the topic of procedural generation for generating the map for the game. I have consulted various implementations and tested one of my own using techniques from what I have found in the material.





The Scrum methodology is employed for the project, 

It consists of a simple set of rules for its' framework that remove speculations, and unpredictability in development by progressing in bursts of short time period, known as sprint cycles. This allows for achievement of tangible results consistently, which feed back to the process as resources for future cycles. 

After the specification and requirements have been formulated for the project. The necessary components of the game were identified, creating a backlog of tasks to be accomplished for the project. They are illustrated below, where these tasks are gradually completed as development progresses with each sprint cycle. 

Backend

Frontend

Design of the website 

Game Mechanics

Map Generation

2D Sprite for tiles

Hex Grid Implementation


The period of each sprint cycle was approximately a week long, and they were managed by ... They served the role of Scrum masters who oversaw sprint cycles to make sure of progress and steer the direction of development.
 









Recovery strategies: Version control

Github is used as the version control tool for the project. This is to prevent accidental loss of progress due to any reason.






Organisation

Weekly meeting with the supervisor, Jane Sinclair, was organised and recorded on Tabula. In each meeting, new progress of development is shared and she will advise on the direction of development.

Frequent meeting between the group members also took place. This is usually at the end of sprint cycles, to discuss about any challenges encountered during implementation and what the next steps in development will be.




Meeting Minutes were 




Meeting Minutes











\subsection*{Team Members Roles}

\subsection*{Development Approach}

\section*{Development}
\subsection*{Front End}
\subsection*{Back End}
\subsection*{Map Generation}

Initially map is generated randomly. Tiles of any type could be placed anywhere.

Procedural generation is then looked into. Introduced the importance of noise functions


Also found map generation through the use of voronoli diagrams using polygons. Several implementations were found and they were looked into as reference. Altered the method slightly in order to fit in with the format of the game world. 

It introduces centroids into the map and determine the property of each tile based on the distance with each centroids.

The method is as follows:

Decide on the number and property of the points

Generate points within the range of the map

Calculate the distance between each tiles and the points




Initially the uniform random function is used to test the method. 

Other random function could be used if necessary

Euclidean distance is used for comparison of distance


Initially testing only included the generation of regions for tiles, i.e areas of dessert, grassland, and rock

Need to extend the method further to include mountain area and/ or water








\section*{Evaluation}
\subsection*{Front End}
\subsection*{Back End}
\subsection*{Map Generation}

Takes the data in json format from the backend- use the data to find the range of q and r determined from the backend to generate a map

Centroids are generated within the range of the map using an uniform random function. Each one will be of a specific type, i.e grass, stone, etc. 

Each tile of the map is compared against each generated centroids in terms of distance, the type of a tile is determined by the proximity with the centroids. 

Depending on the type that have been assigned, further random functions are used to decide if the tile will contain resources of the game. 

The data is then condensed and feeds back to the backend for processing. 



\section*{Conclusion}

\end{document}
