\label{Strategy_Games}

\subsection*{Strategy Games}

For each type of game, there exists a core mechanic. This is a set of actions within the game that prioritise the rules of the game in order to have a greater chance of winning. The core mechanic of strategy games is to plan better than your opponents through utilising the available resources and options in the game to your advantage. Despite of the same core mechanic, there exist many different types of strategy games, real time or turn based, strategy or tactic based, war or empire simulator, etc. 

This project focuses on 4x strategy games. 4x stands for explore, expand, exploit and exterminate. These are the main concepts of the game that formulate different mechanics that intertwine with each other in conforming the core mechanic.


Exploration is about gathering information concerning the game world, such as where the other players are located, where are the points of interest in terms of resources or other strategic advantages. By having knowledge about the terrain and the surrounding, this allows the player to make more informed decisions and better planning.

Expansion makes use of opportunities for scaling up the existing amenities and production. The focus is to take advantage of the increased capacity in the existing mechanism to make more opportunities in the future. This allows the player to be equipped with more resources and information, which enables the player to continue expansion in the future

Exploitation is concerned with grasping the other mechanism of the game, in order to make full use of the systems in place. For many games, this comes in the form of technology tree upgrades that could improve the efficiency of current productions. But it could be anything that could be altered by the player through careful planning, to fully realise the potential of the mechanism in place. 

Extermination is as the name suggests, erasing the existence or significance of other players. This does not necessarily mean that combat mechanics are always required. Alternatives are actions that could render the impact of other players redundant, for example, in ..., there are ... victories that allow a player to win through completing ... And by finishing ... before others, this will win the game, which makes everything that the other players have done insignificant.  

Despite the differences between the 4 concepts, they are all related. For example, without exploration the player will not be able to make the best out of expansion. Exploration make information available to be exploited. Expansion is also a form of exploitation of the game mechanic to extend the current resources. By constantly expanding, the player will be forced to confront other players where elements of extermination comes into play. As there can only be one victor in the end.

Together, the 4 concepts create a rich environment of content for the game, where the different mechanics develop differently as the game progress throughout the 3 stages: start of the game, mid game, and late game.

At the start of the game, the focus is on exploration and using the nearby environment or the resources available to create a foundation. This will be built upon over time, switching the focus onto expansion and exploitation to unlock different components from the mechanics of the game. This allows players to begin their strategy on achieving victory. 


The focus of the mid game is primarily on exploitation, where players begin executing their strategies in full effect through the foundation they have been building from the start of the game. This is also when the players are most likely going to clash with each other from expansion, which could change the focus of some towards extermination.

In the late game, majority of the players will have their strategies well established, such that the focus from the mid game is carried over to the late game as well. A common issue in the late game is the snowball effect, this is when the game becomes a waiting game due to a substantial difference between one player with the others. That one particular player possess such an excess of resources or control over the game world, it becomes only a matter of time for that player to win.

For a lot of strategy games, the late game is the least entertaining part of the game. This is because resources and amenities will have been very well established from the early parts of the game, such that the value of them have lost significance. They become something that players will always have at their disposal, so the focus becomes more about management. Also, the different elements of the game and components of the mechanics would have been fully explored. The lack of content that could be offered at that point is a huge contrast to the early and mid game. 











\begin{comment}



\end{comment}
