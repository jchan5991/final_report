\subsection*{Technology Employed}

JavaScript is used for the front end of the game, it renders the game world and the action that takes place in it within the browser. JavaScript enables for the creation of interactive content and control of the multimedia available from the browser. Its' capabilities from runtime object construction to dynamic script evaluation allows for concise and effective development. Such properties are all ideal for the making of a game. 

Two frameworks of Javascript in particular are used for the project, Angular JS and Pixi JS. 


 
AngularJS provides a structural framework for designing dynamic web applications through extending HTML's syntax, such that components of application can be represented with data binding and dependency injection. This allows for a template to be built for the project, where the content can be presented consistently. New components can also be added to the existing structure of the application easily through the utilised template.




Pixi JS provides a framework for rendering graphics with WebGL support or the HTML5’s canvas. It is a better alternative to Adobe Flash that could design quality material effectively and with ease, especially with Flash's decrease of a presence on the web. It achieves this by avoiding the focus on the low level code required to create interactive content. There are no worries of browser or device inconsistencies by allowing for cross platform compatibility and graceful degradation.







Maven is used as the build tool for the game. It provides an uniform build system that aims to provide a build process that is easy to use, that provides crucial information regarding the project, and the capacity to update features of the project with transparency. The uniform system is useful for joining all the components of the back end together to formulate the game world of the project.







Scala is the primary language used to write the back end of the project. Scala incorporates both object oriented and functional programming, allowing the code of complex operations such as recursions or complicated structures to be simplified. Operations in the language are concise and can be combined with ease. It is statically typed which prevents misuses and errors from the collection of operations.This is chosen for the project as it has the capacity to simplify the coding for a lot of the more complex mechanics. Being statically typed also helps to identify any major fault from development during compilation, this reduces the risk of any game breaking bug making through to the final product.  







Insomnia is used to test out the JSON querying from the game server. 









